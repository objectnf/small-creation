\documentclass[algorithm, pgfplots]{styles/cuzbeamer_objnf}

\usepackage[scale=2]{ccicons}
\usepackage{listings, lstfiracode, color}

\setsansfont{Consolas}
\setmonofont{Fira Code}[Contextuals=Alternate]
\setCJKmainfont{方正书宋_GBK}
\setCJKfamilyfont{yasong}{方正特雅宋强国版_GBK}
\newcommand{\yasong}{\CJKfamily{yasong}}

\begin{document}
    \ActivateVerbatimLigatures

    \title{\yasong\textbf{\fontsize{25.0pt}{\baselineskip}\selectfont 112233}}
\subtitle{\yasong\textbf{\fontsize{20.0pt}{\baselineskip}\selectfont 112233}}
\date{\today}
\author{\fontsize{12.0pt}{\baselineskip}\selectfont 112233}
\email{}
    \maketitle

    \section{\yasong\fontsize{12.0pt}{\baselineskip}\selectfont 使用AFL对LibModBus进行测试}

    \begin{fragile}
        \subsection{为什么改掉socket?}
        \begin{block}{\yasong 为什么改掉socket?}
            \begin{enumerate}
                \item AFL等绝大多数Fuzz工具并不提供对TCP、UDP甚至自定义传输层协议的支持。
                \item 建立socket、通过socket读取和写入数据的效率远远低于通过标准输入输出传递数据,这会严重影响Fuzz的速度。
            \end{enumerate}
        \end{block}
        \begin{block}{\yasong 如何改掉socket?}
            \begin{itemize}
                \item 修改程序或库,使其不使用socket而是直接从标准输入接受数据、运行结果回显至标准输出。
                \item 修改操作系统的socket实现,使调用socket函数的程序实际上不走socket而是走标准输入输出。
            \end{itemize}
        \end{block}
    \end{fragile}

    \begin{fragile}
        \subsection{存在的问题?}
        \begin{block}{\yasong 存在的问题?}
            \begin{itemize}
                \item 直接在程序中改掉socket会不会引入其他问题?这些程序问题会不会干扰Fuzz过程?
                \item 不同的网络程序或库进行Fuzz,修改的方式不一样,对大量网络程序或库进行Fuzz时的时间消耗?
                \item 难道直接改掉syscall?可行性?其他使用socket做进程间通信的程序怎么办?
            \end{itemize}
            \ \\
            \begin{center}
                \yasong 是否有两全其美的办法,既不修改源代码,又不误伤其他程序?
            \end{center}
        \end{block}
    \end{fragile}

    \begin{fragile}
        \subsection{LD\_PRELOAD}
        \begin{block}{LD\_PRELOAD}
            \ \\
            LD\_PRELOAD是Linux系统中的一个环境变量,允许定义在程序运行前优先加载的动态链接库。通过这个环境变量,我们可以有选择地载入{\yasong 包含相同函数}但{\yasong 函数的实现不同}的动态链接库,也可以\textcolor{red}{\yasong 替换掉功能正常的库函数},以达到某些目的。
        \end{block}
        \begin{block}{Preeny}
            \ \\
            \textcolor{blue}{https://github.com/zardus/preeny} \\
            是AFL官方推荐的项目,提供了许多互相独立的程序库。利用LD\_PRELOAD替换掉不同的系统函数(如alarm、fork、socket、rand等),使我们对程序的调试和修改更加方便。\textcolor{red}{\yasong 在Fuzz LibModBus时,我们主要用到其中的desock模块。}
        \end{block}
    \end{fragile}

    \begin{fragile}
        \subsection{编译Preeny}
        \begin{block}{\yasong 编译Preeny}
            \ \\
            可以很方便的下载Preeny的源码并编译。由于desock库依赖logging库记录日志,因此这两个库都需要编译:
            \begin{minted}{bash}
git clone https://github.com/zardus/preeny.git
cd preeny
make
            \end{minted}
            随后即可通过LD\_PRELOAD载入编译生成的动态链接库(.so)文件,命令如下:
            \begin{minted}{bash}
LD_PRELOAD=../libmodbus-afl-dist/lib/libmodbus.so:../../Fuzzer/preeny-dist/desock.so:../../Fuzzer/preeny-dist/logging.so ./slave
            \end{minted}
            即可启动socket被重定向至标准输入输出的slave。
        \end{block}
    \end{fragile}
\end{document}
