\documentclass[algorithm,pgfplots]{styles/cuzbeamer}

\usepackage[scale=2]{ccicons}

\begin{document}
    \title{\yasong\textbf{\fontsize{25.0pt}{\baselineskip}\selectfont 112233}}
\subtitle{\yasong\textbf{\fontsize{20.0pt}{\baselineskip}\selectfont 112233}}
\date{\today}
\author{\fontsize{12.0pt}{\baselineskip}\selectfont 112233}
\email{}
    \maketitle

    \begin{standout}[第X章]
        章节名称\\
        Chapter Name
    \end{standout}

    \begin{frame}{大纲}
        \setbeamertemplate{section in toc}[sections numbered]
        \tableofcontents
    \end{frame}

    \section{简介/前情回顾}

    \begin{frame}
        \begin{block}{引用}
            \begin{leftbar}
                “不应当急于求成,应当去熟悉自己的研究对象,锲而不舍,时间会成全一切。凡事开始最难,然而更难的是何以善终。”\par
                \raggedleft ------莎士比亚
            \end{leftbar}
            \begin{leftbar}
                “其实地上本没有路,走的人多了,也便成了路。”\par
                \raggedleft ------鲁迅
            \end{leftbar}
            \begin{leftbar}
                “这个世界是通的。”\par
                \raggedleft ------无名禅师
            \end{leftbar}
        \end{block}
    \end{frame}

    \section{进入正题}

    \begin{frame}
        \begin{columns}[T,onlytextwidth]
            \column{0.33\textwidth}
            \begin{block}{普通标题}
                \begin{itemize}
                    \item 普通内容
                    \item 普通内容
                    \item 普通内容
                \end{itemize}
            \end{block}
            \metroset{block=fill}
            \begin{block}{普通标题}
                \begin{itemize}
                    \item 普通内容
                    \item 普通内容
                    \item 普通内容
                \end{itemize}
            \end{block}
            \column{0.33\textwidth}
            \begin{alertblock}{警告标题}
                \begin{enumerate}
                    \item 警告内容
                    \item 警告内容
                    \item 警告内容
                \end{enumerate}
            \end{alertblock}
            \metroset{block=fill}
            \begin{alertblock}{警告标题}
                \begin{enumerate}
                    \item 警告内容
                    \item 警告内容
                    \item 警告内容
                \end{enumerate}
            \end{alertblock}
            \column{0.33\textwidth}
            \begin{exampleblock}{例子标题}
                \begin{description}
                    \item[描述定义] 描述内容
                    \item[描述定义] 描述内容
                    \item[描述定义] 描述内容
                \end{description}
            \end{exampleblock}
            \metroset{block=fill}
            \begin{exampleblock}{例子标题}
                \begin{description}
                    \item[描述定义] 描述内容
                    \item[描述定义] 描述内容
                    \item[描述定义] 描述内容
                \end{description}
            \end{exampleblock}
        \end{columns}
    \end{frame}

    \begin{fragile}
        \bicolumns{
            \begin{block}{图片}
                \begin{center}
                    \includegraphics[width=0.65\textwidth]{cuzlogo}\\
                    \includegraphics[width=0.65\textwidth]{cuzlogo-dark}\\
                    \includegraphics[width=0.65\textwidth]{cuzlogo-light}\\
                    \includegraphics[width=0.65\textwidth]{cuzlogo-brown}\\
                \end{center}
            \end{block}
        }{
        \begin{block}{表格}
            \begin{table}\small
                % \caption{样表}
                % \caption*{
                %   表例里甚至可以加表格:
                %   \begin{tabular}{l l}
                %     \toprule
                %     第一项 & 解释一 \\
                %     \bottomrule
                %   \end{tabular}
                % }
                \begin{tabular}{llr}
                    \toprule
                    \multicolumn{2}{c}{项目} & \multirow{2}{*}{价格 (\$)} \\
                    \cmidrule(r){1-2}
                    动物                    & 描述 &  \\
                    \midrule
                    \multirow{2}{*}{蚋蚊}   & 每克 & 13.65 \\
                                            & 每只 & 0.01 \\
                    角马                    & 标本 & 92.50 \\
                    鸸鹋                    & 标本 & 33.33 \\
                    犰狳                    & 冷冻 & 8.99 \\
                    \bottomrule
                \end{tabular}
            \end{table}
        \end{block}
        }
    \end{fragile}

    \begin{fragile}[]
        \bicolumns[0.382]{
        }{
            \begin{block}{代码}
                \begin{minted}{c++}
                    // 西加加
                    #include <iostream>
                    
                    auto main(int argc, char const **argv) {
                        std::cout << "Test" << std::endl;
                        return 0;
                    }
                \end{minted}
                \begin{minted}{java}
                    // 爪哇
                    class Test {
                        public static void main(String[] args) {
                            System.out.println("Test");
                        }
                    }
                \end{minted}
                \begin{minted}{python}
                    # 蟒蛇,测试数学公式:$\int_{-\infty}^{+\infty}{f(x)\mathrm{d}x}$
                    print('Test')
                \end{minted}
            \end{block}
        }
    \end{fragile}

    \begin{fragile}[]
        \bicolumns[0.382]{
        }{
            \begin{block}{算法}
                \begin{algorithmx}[alg:euclid]{Euclid算法}
                    \Procedure{Euclid}{$a,b$}\Comment{$a$与$b$的最大公约数}
                    \State $r\gets a\bmod b$
                    \While{$r\neq 0$}\Comment{若$r$为0则可跳出循环返回答案}
                    \State $a\gets b$
                    \State $b\gets r$
                    \State $r\gets a\bmod b$
                    \EndWhile\label{euclidendwhile}
                    \State \textbf{return} $b$\Comment{最大公约数为$b$}
                    \EndProcedure
                \end{algorithmx}
            \end{block}
        }
    \end{fragile}

    \begin{frame}
        \begin{block}{参考文献}
            \begin{itemize}
                \item 可按如下方式引用参考文献:\cite{cheng2007dahua,cooper2007face,freeman2004head,gamma1994design,graham2004hackers,norman2013design,nystrom2014game,williams2014non}
            \end{itemize}
        \end{block}
    \end{frame}

    \begin{frame}
        \begin{block}{类图}
            \begin{center}
                \begin{tikzpicture}
                    \umlemptyclass[x=0,y=1,type=interface]{Vehicle} %
                    \umlemptyclass[x=1,y=-2]{Bicycle} \umlemptyclass[x=-1,y=-2]{Car} %
                    \umlemptyclass[x=-4,y=-1]{Tyre} \umlemptyclass[x=-4,y=-3]{Engine} %
                    \umlemptyclass[x=-1,y=-4]{SUV} \umlemptyclass[x=4,y=-2]{Student} %
                    \umlemptyclass[x=4,y=0]{School} \umlemptyclass[x=4,y=-4]{ID} %
                    
                    \umlimpl[geometry=|-|]{Car}{Vehicle} %
                    \umlimpl[geometry=|-|]{Bicycle}{Vehicle} %
                    \umlcompo[geometry=-|-]{Car}{Tyre} %
                    \umlcompo[geometry=-|-]{Car}{Engine} %
                    \umlinherit[geometry=--]{SUV}{Car} %
                    \umlaggreg[geometry=|-]{School}{Student} %
                    \umldep[geometry=--]{Student}{Bicycle} %
                    \umlassoc[geometry=--]{Student}{ID} %
                \end{tikzpicture}
            \end{center}
        \end{block}
    \end{frame}

    \begin{fragile}[]
        \begin{block}{状态图}
            \begin{center}
                \begin{tikzpicture}[ >=stealth, ->, thick, black!50 ]
                    \matrix[row sep=20mm, column sep=22mm] { %
                        % First row:
                        & \node (have-money) [terminal] {金额充足}; & \\ %
                        % Second row:
                        \node (sold-out) [terminal] {糖果售罄}; & %
                        \node (no-money) [terminal] {金额不足}; & %
                        \node (sold) [terminal] {售出糖果}; \\ %
                    }; %
                    \begin{scope}[every node/.style={midway,auto,font=\scriptsize}]
                        \draw (have-money) to [bend left] node [align=center]
                        {\mintinline{java}{退回}\\\mintinline{java}{25分钱}} (no-money); %
                        \draw (no-money) to [bend left] node [align=center]
                        {\mintinline{java}{投入}\\\mintinline{java}{25分钱}} (have-money); %
                        \draw (have-money) to [bend left] node [align=center]
                        {\mintinline{java}{转动曲柄}} (sold); %
                        \draw (sold) -- node [align=center] %
                        {\mintinline{java}{发放糖果}\\\mintinline{java}{糖果数>0}} (no-money); %
                        \draw (sold) to [bend left] node [align=center] %
                        {\mintinline{java}{发放糖果}\\\mintinline{java}{糖果数=0}} (sold-out); %
                    \end{scope}
                \end{tikzpicture}
            \end{center}
        \end{block}
    \end{fragile}

    \begin{frame}
        \begin{block}{折线图}
            \begin{figure}
                \begin{tikzpicture}
                    \begin{axis}[
                        mlineplot,
                        width=0.9\textwidth,
                        height=6cm,
                    ]
                    \addplot {sin(deg(x))};
                    \addplot+[samples=100] {sin(deg(2*x))};
                    \end{axis}
                \end{tikzpicture}
            \end{figure}
        \end{block}
    \end{frame}

    \begin{frame}
        \begin{block}{柱状图}
            \begin{figure}
                \begin{tikzpicture}
                    \begin{axis}[
                        mbarplot,
                        xlabel={横},
                        ylabel={纵},
                        width=0.9\textwidth,
                        height=6cm,
                    ]
                    \addplot plot coordinates {(1, 20) (2, 25) (3, 22.4) (4, 12.4)};
                    \addplot plot coordinates {(1, 18) (2, 24) (3, 23.5) (4, 13.2)};
                    \addplot plot coordinates {(1, 10) (2, 19) (3, 25) (4, 15.2)};
                    \legend{甲, 乙, 丙}
                    \end{axis}
                \end{tikzpicture}
            \end{figure}
        \end{block}
    \end{frame}

    \section{小结}

    \begin{frame}{}
        \begin{leftbar}
            您可通过如下链接之一获取本\texttt{cuzbeamer}模板与相应的示例文件:
            \par
            \begin{center}
                \href{https://github.com/xiehao/cuzbeamer}{https://github.com/xiehao/cuzbeamer}\par
                \href{https://bitbucket.org/xiehao/cuzbeamer}{https://bitbucket.org/xiehao/cuzbeamer}
            \end{center}
            本模板本身遵循如下
            \href{http://creativecommons.org/licenses/by-sa/4.0/}{署名-相同方式
            共享4.0国际许可证(CC BY-SA 4.0)}:
            \begin{center}
                \ccbysa
            \end{center}
        \end{leftbar}
    \end{frame}

    \begin{standout}[]
        感谢聆听!
    \end{standout}

    \appendix

    \section{附录}

    \begin{frame}[allowframebreaks]{参考文献}
        \bibliography{references}
        \bibliographystyle{unsrt}
    \end{frame}

\end{document}